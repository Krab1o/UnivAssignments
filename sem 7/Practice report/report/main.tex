\documentclass[pract]{SCWorks}
% Тип обучения (одно из значений):
%    bachelor   - бакалавриат (по умолчанию)
%    spec       - специальность
%    master     - магистратура
% Форма обучения (одно из значений):
%    och        - очное (по умолчанию)
%    zaoch      - заочное
% Тип работы (одно из значений):
%    coursework - курсовая работа (по умолчанию)
%    referat    - реферат
%  * otchet     - универсальный отчет
%  * nirjournal - журнал НИР
%  * digital    - итоговая работа для цифровой кафдры
%    diploma    - дипломная работа
%    pract      - отчет о научно-исследовательской работе
%    autoref    - автореферат выпускной работы
%    assignment - задание на выпускную квалификационную работу
%    review     - отзыв руководителя
%    critique   - рецензия на выпускную работу
% Включение шрифта
%    times      - включение шрифта Times New Roman (если установлен)
%                 по умолчанию выключен
\usepackage{preamble}
% \captionsetup[figure]{font= normalsize, labelfont=normalsize}
\renewcommand\theFancyVerbLine{\small\arabic{FancyVerbLine}}

\begin{document}

% Кафедра (в родительном падеже)
\chair{математической кибернетики и компьютерных наук}

% Тема работ
\title{Курс разработки Backend Web-приложения с DevOps принципами}

% Курс
\course{3}

% Группа
\group{351}

% Факультет (в родительном падеже) (по умолчанию "факультета КНиИТ")
% \department{факультета КНиИТ}

% Специальность/направление код - наименование
% \napravlenie{02.03.02 "--- Фундаментальная информатика и информационные технологии}
% \napravlenie{02.03.01 "--- Математическое обеспечение и администрирование информационных систем}
% \napravlenie{09.03.01 "--- Информатика и вычислительная техника}
\napravlenie{09.03.04 "--- Программная инженерия}
% \napravlenie{10.05.01 "--- Компьютерная безопасность}

% Для студентки. Для работы студента следующая команда не нужна.
% \studenttitle{Студентки}

% Фамилия, имя, отчество в родительном падеже
\author{Устюшина Богдана Антоновича}

% Заведующий кафедрой 
\chtitle{доцент, к.\,ф.-м.\,н.}
\chname{С.\,В.\,Миронов}

% Руководитель ДПП ПП для цифровой кафедры (перекрывает заведующего кафедры)
% \chpretitle{
%     заведующий кафедрой математических основ информатики и олимпиадного\\
%     программирования на базе МАОУ <<Ф"=Т лицей №1>>
% }
%\chtitle{г. Саратов, к.\,ф.-м.\,н., доцент}
% \chname{Кондратова\, Ю.\,Н.}

% Научный руководитель (для реферата преподаватель проверяющий работу)
\satitle{ст. преп., к.\,ф.-м.\,н.} %должность, степень, звание
\saname{М.\,И.\,Сафрончик}

% Руководитель практики от организации (руководитель для цифровой кафедры)
\patitle{доцент, к.\,ф.-м.\,н.}
\paname{С.\,В.\,Миронов}

% Руководитель НИР
\nirtitle{доцент, к.\,п.\,н.} % степень, звание
\nirname{В.\,А.\,Векслер}

% Семестр (только для практики, для остальных типов работ не используется)
\term{6}

% Наименование практики (только для практики, для остальных типов работ не
% используется)
\practtype{производственная}

% Продолжительность практики (количество недель) (только для практики, для
% остальных типов работ не используется)
\duration{4}

% Даты начала и окончания практики (только для практики, для остальных типов
% работ не используется)
\practStart{01.08.2024}
\practFinish{28.08.2024}

% Год выполнения отчета
\date{2024}

\maketitle

% Включение нумерации рисунков, формул и таблиц по разделам (по умолчанию -
% нумерация сквозная) (допускается оба вида нумерации)
% \secNumbering

% \tableofcontents

% Раздел "Обозначения и сокращения". Может отсутствовать в работе
% \abbreviations
% \begin{description}
%     \item ... "--- ...
%     \item ... "--- ...
% \end{description}

% Раздел "Определения". Может отсутствовать в работе
% \definitions

% Раздел "Определения, обозначения и сокращения". Может отсутствовать в работе.
% Если присутствует, то заменяет собой разделы "Обозначения и сокращения" и
% "Определения"
% \defabbr

\sloppy

\intro{}

Практика проходила на базе предприятия ООО <<ПрофСофт>> и заключалась в 
прохождении курса по Backend- и DevOps-разработке. 

В рамках курса велась над учебным проектом <<Анекдоты>>, в рамках которого
показывались лучшие практики в разработке серверной части приложений, а также
техники по поддержке и настройке CI/CD на проекте.

В процессе разработки использовались язык программирования PHP и
объектно-реляционная система управления базами данных PostgreSQL. Также в
рамках работы над DevOps задачами активно использовалась YAML-нотация для
написания настроечных файлов для pipeline на GitLab и написания 
выполняемых в рамках автоматизации работ.

Целью практики была разработка backend-части приложения, а также 
автоматизация его деплоя на GitLab.

В рамках производственной практики должны были быть решены следующие задачи:
\begin{enumerate}
    \item Построение схемы базы данных
    \item Создание базовых CRUD для backend-приложения
    \item Создание авторизации на основе концепции access- и refresh-токенов
    \item Создание docker-compose и gitlab-ci файлов для настройки
    контейнеризации и автоматизации запуска приложения  
\end{enumerate}

\section{Описание проекта}

Современная веб-разработка является одной из самых динамично развивающихся 
областей информационных технологий. С каждым годом всё большее количество
компаний и организаций переносит свои сервисы и системы в онлайн-среду,
что требует создания надежных, масштабируемых и безопасных веб-приложений.
Особое внимание уделяется разработке серверной части (Backend), 
которая обеспечивает связь с базами данных, обработку запросов 
пользователей, а также взаимодействие с различными внешними системами. 
Веб-приложения, использующие язык программирования PHP и фреймворк 
Symfony, предоставляют широкие возможности для создания мощных и гибких
систем, особенно в сочетании с такими современными технологиями, как 
контейнеризация с Docker и использование веб-сервера Nginx. Это делает 
тему разработки серверной части веб-приложений крайне актуальной в 
контексте повышения производительности, гибкости и безопасности
веб-систем.

В данной работе будет рассматриваться процесс разработки серверной
части веб-приложения для обмена анекдотами на базе PHP и фреймворка 
Symfony. В процессе работы будет рассмотрено, как с помощью Docker
создать контейнеризированное окружение для разработки и развертывания
приложения, как настроить веб-сервер Nginx для обработки запросов, а также
как тестировать API с помощью Postman и документировать его с использованием
Swagger. Также будут освещены основные принципы работы с архитектурой 
MVC на Symfony и взаимодействие с базами данных через ORM Doctrine. 
Особое внимание будет уделено выбору и настройке средств разработки, 
таких как PHPStorm, а также сравнению PHP с другими популярными языками 
для создания Backend-приложений. Важной частью работы станет анализ 
производительности и безопасности разрабатываемого приложения в
контейнеризированной среде.

Целью производственной практики является приобретение навыков разработки 
серверной части веб-приложений с использованием современных инструментов 
и технологий, таких как PHP, Symfony, Docker, Nginx, Postman и Swagger. 
Практика направлена на углубленное изучение архитектуры веб-приложений, 
принципов работы серверной части и интеграции различных инструментов 
для разработки, тестирования и развертывания приложений.

Кроме того, значительное внимание будет уделено изучению особенностей
контейнеризации и управления окружением разработки с помощью Docker,
что позволит улучшить навыки работы с современными DevOps-технологиями.

\section{Выполненные в рамках проекта задачи}
\subsection{Построение схемы базы данных}
\subsubsection{Постановка задачи}
Построить схему базы данных для данной предметной области (анекдоты) и 
представить её в виде ER-диаграммы
\subsubsection{Решение}

\subsection{Создание базовых CRUD для Backend-приложения}
\subsubsection{Постановка задачи}
Построить CRUD (API) PHP-приложения на фреймворке Symfony для функционирования
Backend-приложения. 
\subsubsection{Решение}

\subsection{Создание авторизации на основе концепции access- и refresh-токенов}
\subsubsection{Постановка задачи}
Создать механизм авторизации с помощью access- и refresh-токенов.
\subsubsection{Решение}
\subsection{Создание docker-compose и gitlab-ci файлов для настройки
    контейнеризации и автоматизации запуска приложения}
\subsubsection{Постановка задачи}
\subsubsection{Решение}

\conclusion

В ходе выполнения данной работы были изучены современные инструменты и 
технологии для разработки серверной части веб-приложений. В частности, 
были рассмотрены особенности работы с языком PHP и фреймворком Symfony, 
а также разработана архитектура приложения на основе MVC. 

Настроено контейнеризированное окружение с использованием Docker и веб-сервера 
Nginx для обеспечения эффективной и стабильной работы приложения. Особое 
внимание уделялось тестированию RESTful API с помощью Postman и 
документированию его с использованием Swagger. Также проведен анализ 
преимуществ и недостатков применяемых технологий в контексте разработки 
масштабируемых и безопасных веб-приложений. 

Перспективы применения результатов данной работы достаточно широки. 

Во-первых, разработанное приложение может быть использовано как основа 
для дальнейшего развития и расширения функциональности, например, для 
добавления новых типов контента или внедрения системы рекомендаций на 
основе предпочтений пользователей. 

Во-вторых, контейнеризация с использованием Docker позволяет легко 
масштабировать приложение и развертывать его в различных окружениях, 
что важно для гибкости и мобильности современных веб-сервисов. 

Применение Nginx как веб-сервера повышает производительность и надежность 
работы приложения при обработке большого количества запросов, что открывает 
возможности для использования данного решения в высоконагруженных системах. 
Кроме того, навыки тестирования и документирования API с Postman и Swagger 
могут быть применены в разработке других проектов, требующих четкой и 
структурированной документации интерфейсов.

Таким образом, созданный проект достигнул поставленных задач школы ProfSoft,
цель производственной практики была достигнута, а все поставленные в ходе
практики задачи решены.

\end{document}
