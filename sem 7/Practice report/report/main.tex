\documentclass[pract]{SCWorks}
% Тип обучения (одно из значений):
%    bachelor   - бакалавриат (по умолчанию)
%    spec       - специальность
%    master     - магистратура
% Форма обучения (одно из значений):
%    och        - очное (по умолчанию)
%    zaoch      - заочное
% Тип работы (одно из значений):
%    coursework - курсовая работа (по умолчанию)
%    referat    - реферат
%  * otchet     - универсальный отчет
%  * nirjournal - журнал НИР
%  * digital    - итоговая работа для цифровой кафдры
%    diploma    - дипломная работа
%    pract      - отчет о научно-исследовательской работе
%    autoref    - автореферат выпускной работы
%    assignment - задание на выпускную квалификационную работу
%    review     - отзыв руководителя
%    critique   - рецензия на выпускную работу
% Включение шрифта
%    times      - включение шрифта Times New Roman (если установлен)
%                 по умолчанию выключен
\usepackage{preamble}
% \captionsetup[figure]{font= normalsize, labelfont=normalsize}
\renewcommand\theFancyVerbLine{\small\arabic{FancyVerbLine}}

\begin{document}

% Кафедра (в родительном падеже)
\chair{математической кибернетики и компьютерных наук}

% Тема работ
\title{Курс разработки backend web-приложения с DevOps принципами}

% Курс
\course{3}

% Группа
\group{351}

% Факультет (в родительном падеже) (по умолчанию "факультета КНиИТ")
% \department{факультета КНиИТ}

% Специальность/направление код - наименование
% \napravlenie{02.03.02 "--- Фундаментальная информатика и информационные технологии}
% \napravlenie{02.03.01 "--- Математическое обеспечение и администрирование информационных систем}
% \napravlenie{09.03.01 "--- Информатика и вычислительная техника}
\napravlenie{09.03.04 "--- Программная инженерия}
% \napravlenie{10.05.01 "--- Компьютерная безопасность}

% Для студентки. Для работы студента следующая команда не нужна.
% \studenttitle{Студентки}

% Фамилия, имя, отчество в родительном падеже
\author{Устюшина Богдана Антоновича}

% Заведующий кафедрой 
\chtitle{доцент, к.\,ф.-м.\,н.}
\chname{С.\,В.\,Миронов}

% Руководитель ДПП ПП для цифровой кафедры (перекрывает заведующего кафедры)
% \chpretitle{
%     заведующий кафедрой математических основ информатики и олимпиадного\\
%     программирования на базе МАОУ <<Ф"=Т лицей №1>>
% }
%\chtitle{г. Саратов, к.\,ф.-м.\,н., доцент}
% \chname{Кондратова\, Ю.\,Н.}

% Научный руководитель (для реферата преподаватель проверяющий работу)
\satitle{ст. преп., к.\,ф.-м.\,н.} %должность, степень, звание
\saname{М.\,И.\,Сафрончик}

% Руководитель практики от организации (руководитель для цифровой кафедры)
\patitle{доцент, к.\,ф.-м.\,н.}
\paname{С.\,В.\,Миронов}

% Руководитель НИР
\nirtitle{доцент, к.\,п.\,н.} % степень, звание
\nirname{В.\,А.\,Векслер}

% Семестр (только для практики, для остальных типов работ не используется)
\term{6}

% Наименование практики (только для практики, для остальных типов работ не
% используется)
\practtype{производственная}

% Продолжительность практики (количество недель) (только для практики, для
% остальных типов работ не используется)
\duration{4}

% Даты начала и окончания практики (только для практики, для остальных типов
% работ не используется)
\practStart{01.08.2024}
\practFinish{28.08.2024}

% Год выполнения отчета
\date{2024}

\maketitle

% Включение нумерации рисунков, формул и таблиц по разделам (по умолчанию -
% нумерация сквозная) (допускается оба вида нумерации)
% \secNumbering

% \tableofcontents

% Раздел "Обозначения и сокращения". Может отсутствовать в работе
% \abbreviations
% \begin{description}
%     \item ... "--- ...
%     \item ... "--- ...
% \end{description}

% Раздел "Определения". Может отсутствовать в работе
% \definitions

% Раздел "Определения, обозначения и сокращения". Может отсутствовать в работе.
% Если присутствует, то заменяет собой разделы "Обозначения и сокращения" и
% "Определения"
% \defabbr

\sloppy

\intro{}

Практика проходила на базе предприятия ООО <<ПрофСофт>> и заключалась в 
прохождении курса по Backend- и DevOps-разработке. 

В рамках курса велась над учебным проектом <<Анекдоты>>, в рамках которого
показывались лучшие практики в разработке серверной части приложений, а также
техники по поддержке и настройке CI/CD на проекте.

В процессе разработки использовались язык программирования PHP и
объектно-реляционная система управления базами данных PostgreSQL. Также в
рамках работы над DevOps задачами активно использовалась YAML-нотация для
написания настроечных файлов для pipeline на GitLab и написания 
выполняемых в рамках автоматизации работ.

Целью практики была разработка backend-части приложения, а также 
автоматизация его деплоя на GitLab.

В рамках производственной практики должны были быть решены следующие задачи:
\begin{enumerate}
    \item Построение схемы базы данных
    \item Создание базовых CRUD для backend-приложения
    \item Создание авторизации на основе концепции access- и refresh-токенов
    \item Создание docker-compose и gitlab-ci файлов для настройки
    контейнеризации и автоматизации запуска приложения  
\end{enumerate}

\section{Описание проекта}

Backend-приложение в рамках школы ProfSoft предназначено для получения
признаков 

\conclusion

Тест

\end{document}
