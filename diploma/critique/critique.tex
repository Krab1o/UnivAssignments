\documentclass[critique]{SCWorks}
% Тип обучения (одно из значений):
%    bachelor   - бакалавриат (по умолчанию)
%    spec       - специальность
%    master     - магистратура
% Форма обучения (одно из значений):
%    och        - очное (по умолчанию)
%    zaoch      - заочное
% Тип работы (одно из значений):
%    coursework - курсовая работа (по умолчанию)
%    referat    - реферат
%  * otchet     - универсальный отчет
%  * nirjournal - журнал НИР
%  * digital    - итоговая работа для цифровой кафдры
%    diploma    - дипломная работа
%    pract      - отчет о научно-исследовательской работе
%    autoref    - автореферат выпускной работы
%    assignment - задание на выпускную квалификационную работу
%    review     - отзыв руководителя
%    critique   - рецензия на выпускную работу
% Включение шрифта
%    times      - включение шрифта Times New Roman (если установлен)
%                 по умолчанию выключен
\usepackage{preamble}
% \captionsetup[figure]{font= normalsize, labelfont=normalsize}
\renewcommand\theFancyVerbLine{\small\arabic{FancyVerbLine}}

\begin{document}

% Кафедра (в родительном падеже)
\chair{математической кибернетики и компьютерных наук}

% Тема работ
\title{Разработка web-приложения для организации
процесса уборки мусора}

% Курс
\course{4}

% Группа
\group{451}

% Факультет (в родительном падеже) (по умолчанию "факультета КНиИТ")
% \department{факультета КНиИТ}

% Специальность/направление код - наименование
% \napravlenie{02.03.02 "--- Фундаментальная информатика и информационные технологии}
% \napravlenie{02.03.01 "--- Математическое обеспечение и администрирование информационных систем}
% \napravlenie{09.03.01 "--- Информатика и вычислительная техника}
\napravlenie{09.03.04 "--- Программная инженерия}
% \napravlenie{10.05.01 "--- Компьютерная безопасность}

% Для студентки. Для работы студента следующая команда не нужна.
% \studenttitle{Студентки}

% Фамилия, имя, отчество в родительном падеже
\author{Устюшина Богдана Антоновича}

% Заведующий кафедрой 
\chtitle{доцент, к.\,ф.-м.\,н.}
\chname{С.\,В.\,Миронов}

% Руководитель ДПП ПП для цифровой кафедры (перекрывает заведующего кафедры)
% \chpretitle{
%     заведующий кафедрой математических основ информатики и олимпиадного\\
%     программирования на базе МАОУ <<Ф"=Т лицей №1>>
% }
% \chtitle{г. Саратов, к.\,ф.-м.\,н., доцент}
% \chname{Кондратова\, Ю.\,Н.}

% Научный руководитель (для реферата преподаватель проверяющий работу)
\satitle{доцент, к.\,ф.-м.\,н.} %должность, степень, звание
\saname{А.\,С.\,Иванов}

% Руководитель практики от организации (руководитель для цифровой кафедры)
\patitle{доцент, к.\,ф.-м.\,н.}
\paname{С.\,В.\,Миронов}

% Руководитель НИР
\nirtitle{доцент, к.\,п.\,н.} % степень, звание
\nirname{В.\,А.\,Векслер}

% Семестр (только для практики, для остальных типов работ не используется)
\term{2}

% Наименование практики (только для практики, для остальных типов работ не
% используется)
\practtype{учебная}

% Продолжительность практики (количество недель) (только для практики, для
% остальных типов работ не используется)
\duration{2}

% Даты начала и окончания практики (только для практики, для остальных типов
% работ не используется)
\practStart{01.07.2022}
\practFinish{13.01.2023}

% Год выполнения отчета
\date{2025}

\maketitle

% Включение нумерации рисунков, формул и таблиц по разделам (по умолчанию -
% нумерация сквозная) (допускается оба вида нумерации)
% \secNumbering

% 12\tableofcontents

% Раздел "Обозначения и сокращения". Может отсутствовать в работе
% \abbreviations
% \begin{description}
%     \item ... "--- ...
%     \item ... "--- ...
% \end{description}

% Раздел "Определения". Может отсутствовать в работе
% \definitions

% Раздел "Определения, обозначения и сокращения". Может отсутствовать в работе.
% Если присутствует, то заменяет собой разделы "Обозначения и сокращения" и
% "Определения"
% \defabbr

\sloppy

Выпускная квалификационная работа Устюшина Богдана Антоновича посвящена 
разработке web-приложения, направленного на повышение эффективности 
взаимодействия граждан и служб, ответственных за санитарное состояние 
городской среды. Учитывая актуальность экологических проблем и необходимость 
вовлечения населения в процессы благоустройства, тема работы является 
своевременной и социально значимой.

Структура работы включает введение, три главы, заключение, список 
использованных источников и приложения. Во введении автор обосновывает 
актуальность выбранной темы, формулирует цель и задачи. Первая глава 
содержит подробный обзор технологий, применённых в проекте, включая язык 
программирования Go, фреймворк React, систему управления базами данных 
PostgreSQL, средства контейнеризации Docker и инструменты разработки. Во 
второй главе рассматриваются этапы анализа и проектирования системы, 
описывается архитектура приложения, реализованная с использованием шаблона 
controller-service-repository. Третья глава демонстрирует применение 
разработанного программного продукта: описан пользовательский сценарий, 
приведены примеры регистрации, создания и выполнения событий. Заключение 
подводит итоги проделанной работы. В приложениях представлены структура 
проекта и листинг кода, снабжённого комментариями, а также полный исходный код
приложения.

Работа отличается хорошей структурой и глубоким практическим содержанием. 
Выбор технологий обусловлен необходимостью создания масштабируемого, 
отказоустойчивого и легко поддерживаемого приложения. Практическая 
реализация включает создание полноценного backend и frontend компонентов,
использование JWT для авторизации, реализацию роли модератора и 
пользовательских сценариев, а также контейнеризацию с применением 
Docker Compose.

К достоинствам работы следует отнести:
\begin{itemize}
\item тщательное обоснование выбора технологического стека;
\item реализацию современной архитектуры с разделением на слои;
\item наличие автогенерируемой документации API и продуманной структуры базы данных;
\item внимание к вопросам безопасности и модульности кода.
\end{itemize}
В качестве замечаний можно отметить:
\begin{itemize}
    \item ограниченный пользовательский функционал (например, 
    отсутствие системы уведомлений);
    \item в теоретической части можно было уделить больше внимания 
    аналогичным существующим решениям и провести их сравнительный анализ.
\end{itemize}

Тем не менее, указанные недостатки не являются критическими и не снижают 
общего высокого уровня работы.

В целом, работа «Разработка web-приложения для организации процесса уборки 
мусора» студента Устюшина Богдана Антоновича соответствует заданию и 
выполнена в соответствии с требованиями, предъявляемыми к выпускным 
квалификационным работам.

Считаю, что студент Устюшин Богдан Антонович заслуживает оценки <<Хорошо>> с 
присуждением ему квалификации «Бакалавр» по направлению 09.03.04 — Программная 
инженерия.

\signatureline

\end{document}
